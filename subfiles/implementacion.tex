
\section{Implementación}
En este capítulo expondremos la tecnología usada tanto en el servidor como en la aplicación móvil. También se comentarán las pruebas realizadas.


\subsection{Herramientas de soporte}
En esta sección explicaremos los entornos en los que se trabajo a la hora de desarrollar este proyecto, comentando los fundamento usados en ellos. Esta sección tiene una gran importancia ya que escogerlos bien al inicio nos ayudará a ganar agilidad a la hora de avanzar y cometer menos errores.


\subsubsection{• Eclipse Java EE IDE for Web Developers Version Neon2 }
\subsubsection{• Android Studio}
\subsubsection{• Git}
 Git es un software de control de versiones diseñado por Linus Torvalds, pensando en la eficiencia y la confiabilidad del mantenimiento de versiones de aplicaciones cuando éstas tienen un gran número de archivos de código fuente. Se trata del gestor de versiones moderno
más empleado y cuenta con una gran comunidad de desarrolladores a su alrededor lo
que favorece la integración de Git con múltiples herramientas. Ofreciéndonos 2 opciones para el control de versiones una mediante un interfaz y otra mediante linea de comandos.
\subsubsection{• Apache Maven}
Maven es una herramienta de software para la gestión y construcción de proyectos Java creada por Jason van Zyl.

 Viene con objetivos predefinidos para realizar ciertas tareas comunes en todo proyecto, tales
como la compilación del código o su empaquetado.

 Maven utiliza un Project Object Model (POM) en formato
XML para describir el proyecto de software, sus dependencias con otros módulos
y componentes externos así como el orden de construcción de los elementos.


\subsubsection{• JPA/Hibernate}
\subsubsection{• PostgreSQL}

PostgreSQL es un gestor de bases de datos objecto-relacional  que permite trabajar con grandes cargas de datos consiguiendo una tolerancia alta a errores.
Se decidió usar este gestor porqué tiene una gran adaptabilidad a otros entornos de trabajo lo que ayuda a ganar agilidad y eficiencia. Tambien nos proporciona  el PgAdmin que facilita la gestión y administración de bases de datos ya sea mediante instrucciones SQL o con ayuda de un entorno gráfico. Permite acceder a todas las funcionalidades de la base de datos, consulta, manipulación y gestión de datos.
\subsubsection{• JUnit}

JUnit es el framework de testing para Java más extendido.
Permite la ejecución de clases Java para evaluar el comportamiento de los métodos a testar.



\subsection{Servidor}
\subsubsection{• Engadir funcionalidades a medida nos repositorios de Spring Data}
\subsection{Aplicación móvil Android}
\subsubsection{• Mapas}
\subsubsection{•}
\subsubsection{•}
\subsubsection{•}
\section{Pruebas}
\subsection{Pruebas de unidad}
\subsection{Pruebas de unidad y de integración}


