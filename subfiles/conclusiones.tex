

\section{Trabajo realizado}
Una vez finalizado este Proyecto de Fin de Grado tenemos una aplicación móvil en Android que cumple todos los objetivos marcados al inicio del proyecto y en especial de esta memoria. La aplicación permitirá al usuario gestionar sus grupos, gestionar sus puntos de interés y guardar sus rutas tantos las que hace de manera individual como las que hace de manera conjunta. Esto último lo hace gracias al sistema de localización del GPS.\\


A continuación se muestran las principales características del producto construido: 

\begin{itemize}
\item Registrarse para poder disfrutar de las funcionalidades que ofrecemos y llevar un seguimiento de sus actuaciones en el campo.
\item Guardar puntos estratégicos en el mapa acompañados de un nombre clave y una breve descripción. Esto nos ayudará a recordar puntos para futuras aventuras. 
\item Crear grupos de compañeros y poder añadirlos para compartir rutas.
\item Guardar las rutas seguidas en nuestras aventuras de pesca y de caza, como también la posibilidad de rememorar esas rutas al poder revisarlas gracias a nuestras lista con las rutas seguidas.
\item Y por último la posibilidad de realizar rutas conjuntas con un grupo de amigos que nosotros queramos y así compartir nuestra ubicación en todo momento. Esto nos ayudará en la pesca ya que al conocer la ubicación del resto de integrantes del grupo podríamos socorrerlo si algo le pasa en un sitio de difícil acceso. Como también por tema de seguridad en una jornada de caza ya que si estamos cerca de otro usuario lo veríamos.
\end{itemize}  

\section{Trabajo futuro}
 La aplicación cumple con los objetivos marcados pero como todo proyecto software puede ser ampliado y mejorado. Una vez finalizado este proyecto, se podrían añadir nuevas funcionalidades:



\begin{itemize}
\item Añadir nuevos tipos de PDIs como es el caso de fotografía. Ya que tiene varias semejanzas con la caza y la pesca por los lugares donde se realiza sería una funcionalidad interesante.
\item Como toda actividad que se realiza al aire libre esta condicionada para bien o para mal por condiciones meteorologías, se podrían usar los servicios de OpenWeatherMap. Esto nos ayudaría a planear mejor nuestras jornadas de pesca, caza o fotografía.


\item Otro punto interesante también sería poder añadir a cada PDI una foto asociada a él, lo que ayudaría a acordarse mejor del lugar y ubicarse con precisión.
\end{itemize}
