Intro

\section{Motivación}


\section{Objetivos}

% Objetivos
\begin{itemize}
\item \textbf{Y}
\item \textbf{X}
\end{itemize}


\section{Estructura de la memoria}
La memoria del presente proyecto está estructurada del siguiente modo:

\begin{itemize}
\item \textbf{Introducción} Explica el contexto en el que se enmarca el proyecto, introduce la problemática a tratar y detalla el alcance y objetivos del mismo desde un punto de vista global. También muestra la estructura de la memoria y el plan de trabajo seguido.

\item \textbf{Tecnología} Describe y justifica las principales tecnologías empleadas para desarrollar el objeto del proyecto atendiendo a los requisitos del mismo.

\item \textbf{Conceptos} 

\item \textbf{Proceso de ingeniería} Detalla el proceso de ingeniería: la metodología, la planificación y la gestión del proyecto.

\item \textbf{Desarrollo} Realiza una descripción detallada del análisis, diseño, implementación, pruebas y despliegue del sistema.

\item \textbf{Conclusiones y trabajo futuro} Proporciona una evaluación global del producto obtenido así como futuras líneas de trabajo que se podrían explotar alrededor del proyecto.

\item \textbf{Apéndices} Está compuesto por las siguientes secciones complementarias:
	\begin{itemize}
	\item \textbf{Glosario} Define los términos y acrónimos técnicos empleados en la memoria del proyecto.
	
	\item \textbf{Bibliografía} Recoge los documentación bibliográfica sobre la que se apoya el proyecto.
	\end{itemize}
\end{itemize}


\section{Plan de trabajo}
