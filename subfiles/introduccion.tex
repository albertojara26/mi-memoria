En este primer capitulo de esta memoria se comentaran la motivación, los objetivos de la aplicación y la estructura de esta memoria. 




\section{Motivación}
Dado el creciente uso de los dispositivos móviles en los deportes al aire libre, como en caza y pesca, se aprecia una creciente demanda de aplicaciones que permitan el registro de estas actividades.\\
Por otro lado la seguridad en estas actividades es un dato a tener en cuanta ya que ellas se realizan en lugares de difícil acceso y un tanto peligrosos.\\

Estos motivos nos llevaron a proponer este Trabajo de Fin de Grado.\\

El objetivo de este proyecto será diseñar y construir una herramienta que permita gestionar las actividades a campo abierto como
caza y pesca, permitiendo con geolocalización guardar los puntos de interés, las rutas seguidas,
llevar control de los usuarios presentes, guardar información de los éxitos de las jornadas y poder realizar jornadas de caza y pesca con un seguimiento  total de nuestros compañeros.






 

\section{Objetivos}
El objetivo es desarrollar una aplicación móvil en Android que permita al usuario realizar un seguimiento de sus jornadas tanto de caza como de pesca  y que le permita guardar sus puntos destacadas para poder visitarlos en otras ocasiones. Sin olvidar que el poder monitorizar la jornadas conjuntamente siempre es un punto a favor en el tema de la seguridad en estas actividades.\\ Los principales objetivos serán:

\begin{itemize}
\item Registro de usuarios y adicción de amigos a usuario.
\item El usuario podrá conocer su ubicación en tiempo real en un mapa.
\item El usuario podrá comenzar y parar una jornada.
\item El usuario podrá visualizar la ruta seguida durante la jornada.
\item Administrar puntos clave para el usuario.
\item Clasificación de los puntos de interés según su tipología.
\item Visualizar dichos puntos en un mapa.
\item Gestionar grupos de usuarios participantes en la jornada y poder ver su ubicación en tiempo
real.

\end{itemize}


\section{Estructura de la memoria}
La memoria del presente proyecto está estructurada del siguiente modo:

\begin{itemize}
\item \textbf{Introducción} Contextualiza el proyecto, introduce el tema a tratar y detalla los objetivos del mismo desde un punto de vista global.
 También comenta la estructura de la memoria 

\item \textbf{Tecnología} Describe y justifica las principales tecnologías empleadas para desarrollar el proyecto.



\item \textbf{Proceso de ingeniería} Detallamos el proceso de ingeniería: la metodología empleada,como funcionada, la adaptación de la misma y los participantes.


\item \textbf{Análisis} Comentamos las funcionalidades de la aplicación, comentando mas detenidamente los casos de uso y requisitos. 
\item \textbf{Seguimiento} Comentamos por las fases que pasa el proyecto usando la metodología Scrum.
\item \textbf{Diseño} Describimos la arquitectura del proyecto.
\item \textbf{Implementación} Describiremos aspectos concretos destacados y las pruebas realizadas.





\item \textbf{Conclusiones y trabajo futuro}
Comentamos el producto obtenido y futuros cambios que se podrían realizar.



\end{itemize}


