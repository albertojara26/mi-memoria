El crecimiento del uso de dispositivos móviles en la vida cotidiana a ido en incremento en los últimos años, esto ha hecho que los usuarios también lo quieran usar en deportes al aire libre, como en caza y pesca. En estas actividades normalmente se necesitan recordar puntos clave para poder volver a visitarlos en otras ocasiones o las rutas seguidas para llegar a los mismos. Normalmente estas actividades se realizan en lugares de difícil acceso y un tanto peligrosos, bien por la peligrosa orografía de los accesos como del lugar en si mismo donde se practica.\\

Estos motivos nos llevaron a proponer este Trabajo de Fin de Grado.\\


El objetivo es desarrollar una aplicación móvil en Android que permita al usuario realizar un seguimiento de sus jornadas tanto de caza como de pesca  y que le permita guardar sus puntos destacadas para poder visitarlos en otras ocasiones. Sin olvidar que el poder monitorizar la jornadas conjuntamente siempre es un punto a favor en el tema de la seguridad en estas actividades.\\



Esta aplicación se realizará a través de una aplicación móvil Android la cual será la parte con la que interectuará el usuario y la cual mostrará la información que guardaremos en un servicio web. En ella usaremos otros servicios propios de Android como Google Maps, Firebase y Location. La información mencionada anteriormente será guardada en un servidor con el cual se comunicará la aplicación móvil mediante una API REST, de modo que siempre esté disponible para el usuario.\\



El proyecto seguirá una metodología ágil como es la Scrum, que hará que éste esté dividido en una seria de Sprints. Cada Sprint tendrá las siguientes fases análisis, planificación, diseño, implemtentación y pruebas.\\

Todo lo anteriormente mencionado será recogido en esta memoria y comentado más profundamente.






