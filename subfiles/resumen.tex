	El uso de dispositivos móviles en la vida cotidiana se ha ido incrementando exponencialmente en los últimos años. Como parte de esta tendencia, el uso de los dispositivos móviles se ha extendido a actividades como de ocio como al aire libre, como la pesca o la caza. En estas actividades, normalmente se necesita recordar puntos clave para poder volver a visitarlos en otras ocasiones, o las rutas seguidas para llegar a los mismos. Normalmente estas actividades se realizan en lugares de difícil acceso y un tanto peligrosos, bien por la peligrosa orografía de los accesos como del lugar en si mismo donde se practica.\\

Estos motivos nos llevaron a proponer este Trabajo de Fin de Grado.\\


El objetivo es desarrollar una aplicación móvil Android que permita al usuario realizar un seguimiento de sus jornadas tanto de caza como de pesca  y que le permita guardar sus puntos destacados para poder visitarlos en otras ocasiones. Además, la aplicación permitirá salidas en grupo, en las que cada usuario conocerá la posición de lo demás en todo momento. Poder monitorizar las jornadas conjuntamente siempre es un punto a favor en cuanto a la seguridad en estas actividades.\\



 La aplicación contará con una parte cliente(la propia aplicación nativa Android) y una parte servidor. La aplicación Android será la interfaz con la que que interactuará el usuario y en la que podrá visualizar toda la información. La parte servidora se encargará de almacenar los datos de forma persistente y de implementar los casos de uso. La comunicación entre ambas partes de llevará a cabo mediante servicios Web. En la aplicación móvil usaremos otros servicios propios de Android como Google Maps, Firebase y Location.\\



El proyecto seguirá la metodología ágil Scrum, que hará que éste esté dividido en una serie de Sprints. En cada Sprint  se llevarán a cabo trabajos de análisis, planificación, diseño, implementación y pruebas.\\

Todo lo anteriormente mencionado será recogido en esta memoria y comentado más profundamente.






