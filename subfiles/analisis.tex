\section{Análisis de Requisitos}


A la hora de cumplir los objetivo de este proyecto, se establecieron una serie de requisitos. Estos requisitos fueron obtenidos al  hacer una entrevista con los directores del proyecto. A lo largo del proyecto y siguiendo la metodología, se fueron refinando y añadiendo requisitos. 
El registro del usuario sería el primero para poder darle acceso y comenzar a usarla, ya que sin él no tiene acceso a ninguna funcionalidad de la aplicación.

Después surgió la creación de puntos de interés marcados en un mapa con nombre, descripción y un punto en el mapa con o sin señal GPS clasificados en tipo, caza o pesca.

Creación y almacenamiento de las rutas seguidas por un usuario en sus caminatas por cualquier tipo de terrero.

Adicionalmente, también se completó la posibilidad de que el usuario pueda crear grupos con los usuarios que le parezca oportuno y los integrantes del mismo poder añadir a otros. El resultado de esta funcionalidad es la que permitirá posteriormente crear rutas conjuntas ya para crear una ruta conjunta primero se elige el grupo del que se hará el seguimiento. Una vez elegido se enviarán las invitaciones para participar en él a cada integrante del grupo. Estas invitaciones en el caso de ser aceptadas permitirán al usuario navegar por un mapa y periódicamente se irán realizando actualizaciones de las posiciones del resto de integrantes. Esta funcionalidad tendrá un especial interés en las jornadas de caza ya que el conocer en todo momento la posición del resto de compañeros evita peligros innecesarios. Con todo ello se podrán ver las rutas conjuntas igual que las individuales.
\subsection{Actores}

Los únicos actores que se presentan en la  aplicación son los siguientes:
\begin{itemize}
\item \textbf{Usuario no  autenticado}. Usuario que no está autenticado en la aplicación y que solo
se le permite registrarse en el  sistema o iniciar sesión si ya se registró en otro momento.
\item \textbf{Usuario  autenticado}. Usuario autenticado que puede acceder a todas las funcionalidades
del sistema.
\end{itemize}


\section{Historias de usuario}
Las historias de usuario son los requisitos vistos desde el punto de vista del cliente, es decir, acciones que el usuario llevará a cabo durante el uso del la aplicación. Son la unidad básica detrabajo y se caracteriza por ser:
\begin{itemize}
\item Independientes
 \item Negociables
  \item Estimables
  \item Pequeñas
   \item Tangibles
    
\end{itemize}

Las historias en este proyecto serán las siguientes:


\begin{itemize}
\item Diseño capa intermedia
\item Implementación del servicio Rest
\item Iniciar sesión  
 \item Gestión de puntos de interés (PDI)
  \item Gestión de grupos
  \item Iniciar una ruta individual
   \item Iniciar la ruta compartida
   \item Realización de la memoria.
    
\end{itemize}
 Estas historias anteriormente citadas serán divididas en cada Sprint en pequeñas tareas mas fáciles de manejar. Una tarea es la acción que debe implementar el desarrollador para que se pueda ejecutar parte de la historia, esta está en un lenguaje técnico. Estas tareas reflejan en la mayoría de los casos los denominados casos de uso que comentaremos posteriormente.

\subsection{Casos de uso}
A continuación, en esta sección, se exponen los requisitos funcionales que surgen de los requisitos generales planteados en el punto anterior.
\subsubsection{• Usuario no autenticado}
\begin{itemize}
\item\textbf{ \textit{R1}  Registrarse en la aplicación.}
 El usuario podrá darse de alta en el sistema
introduciendo sus datos en el formulario que se le indican. Una vez registrado se cambia de pantalla para el usuario inserte los datos.

\item \textbf{\textit{R2} Iniciar sesión en la aplicación. }
El usuario ya registrado podrá, con
sus credenciales, autenticarse en el  sistema. Se pedirá el nombre del usuario y su contraseña. Se guardará el estado en el terminal hasta que el usuario decida desconectarse.

Estos dos casos de uso quedan reflejados en la figura \ref{fig:usuario-no-autenticado}.


\end{itemize} 
\begin{figure}
		\centering
		\includegraphics[width=0.75\textwidth] {usuario-no-autenticado.jpg}
		\caption{Casos de uso del actor Usuario No Autenticado }\label{fig:usuario-no-autenticado}
	\end{figure}
	
	
	
\subsubsection{• Usuario  autenticado}

Para el caso que el usuario ya esté autenticado dividiremos en 3 grupos los casos de usos, los cuales podremos ver en la figura \ref{fig:PDI}:
\begin{itemize}
\item \textbf{Gestión de puntos de interés}\\
Aquí se describirán los casos de uso relacionados con la gestión  de la información del los puntos de interés.
\begin{itemize}
\item\textbf{\textit{ R-PDI-1 Guardar Punto De Interés caza}}, el usuario podrá guardar un punto concreto, de caza, asociado a un par de coordenadas pudiendo añadirle un nombre y una descripción.
\item\textit{ \textbf{R-PDI-2 Guardar Punto De Interés pesca}}, el caso de uso es similar al de anterior pero este es para el tipo de pesca.
\item \textbf{\textit{R-PDI-3 Eliminar PDI}}, el usuario podrá seleccionar un punto o una lista de puntos para ser borrados.
\item \textbf{\textit{R-PDI-4 Buscar los PDI}}, permite ver todos los puntos de interés de cada tipo en un mapa y pudiendo clicar en ellos para conocer su nombre y descripción.
\end{itemize} 

\begin{figure}[H]
		\centering
		\includegraphics[width=0.75\textwidth] {PDI.jpg}
		\caption{Casos de uso de gestión de puntos de interés }\label{fig:PDI}
	\end{figure}
	
	
	
\item \textbf{Gestión de grupos}. Ver figura \ref{fig:grupo}
\begin{itemize}
\item\textbf{ \textit{R-G-1 Crear grupo}}, el usuario crea un grupo con nombre único.
\item\textbf{\textit{ R-G-2 Añadir integrantes}}, el usuario busca los usuarios que quiere integrar en el grupo previamente seleccionado.
\item \textbf{\textit{R-G-3 Eliminar integrantes}}, el usuario puede eliminar los integrantes que considere.
\item \textbf{\textit{R-G-4 Ver grupos}}, el sistema lista los grupos a los que el usuario pertenece.
\item \textbf{\textit{R-G-5 Ver integrantes grupo}}, el sistema permitirá ver los integrantes del grupo que el usuario indique, previo listado del caso de uso R-G-4. 

\end{itemize} 

\begin{figure}
		\centering
		\includegraphics[width=0.75\textwidth] {grupo.jpg}
		\caption{Casos de uso de gestión de grupos de usuarios }\label{fig:grupo}
	\end{figure}
	
	
	
\item \textbf{Gestión de rutas}. Ver figura \ref{fig:rutas} 
\begin{itemize}
\item \textbf{\textit{R-R-1 Crear ruta privada}}, el usuario registra la ruta con un nombre.
\begin{itemize}
\item \textbf{\textit{R-R-1.1}} Iniciar ruta, el sistema comienza a guardar las coordenadas por la que el usuario está desplazándose y dibujando la ruta en el mapa. Las coordenadas se irán guardando periódicamente, no solo al finalizar la ruta y visualizarla.
\item\textbf{ \textit{R-R-1.2}} Parar ruta, permite parar la navegación, tanto de guardar las coordenadas como de pintar la ruta seguida.
\item\textbf{ \textit{R-R-1.3}} Reanudar ruta, permite reanudar la navegación.
\item \textbf{\textit{R-R-1.4}} Guardar ruta, el sistema guarda los últimos puntos que quedaban sin actualizar y ejecuta el caso de uso ver ruta en el mapa(R-R-4).
\end{itemize}

\item \textbf{\textit{R-R-2 Crear ruta compartida}}, este caso de uso permite guardar la ruta seguida por el usuario y al mismo tiempo ver la posición del resto de integrantes de un grupo, anteriormente seleccionado, en tiempo real. Este caso de uso también enviaría a los integrantes del grupo una invitación a dicha ruta.
\begin{itemize}
\item \textbf{\textit{R-R-2.1 Iniciar ruta}}, se comienza a guardar y dibujar la ruta en el mapa. Por otra parte se comienza el seguimiento del resto de usuarios que estén también desplazándose, como también una actualización parcial de la ruta seguida en el servidor.
\item \textbf{\textit{R-R-2.2 Parar ruta}}, se para la navegación y se deja de actualizar la posición al resto de usuario de la ruta compartida.
\item \textbf{\textit{R-R-2.3 Reanudar ruta}}, se reanuda la navegación y se  actualizar la posición al resto de usuario de la ruta compartida
\item \textbf{\textit{R-R-2.4 Finalizar ruta}}, se guardan los puntos que faltan de enviar al servidor y se deja de enviar datos al resto de integrantes.
\end{itemize}
\item \textbf{\textit{R-R-3 Listar rutas} }, permite al usuario ver todas las rutas realizadas tanto de manera privada como de manera compartida.
\item \textbf{\textit{R-R-4 Ver ruta en mapa}}, el sistema dibuja en un mapa la ruta seguida y previamente seleccionada.
\item \textbf{\textit{R-R-5 Eliminar ruta}}, permite al usuario borrar de la aplicación la ruta indicada. Dado que el tratamiento de rutas compartidas una vez guardadas es igual a la de rutas individuales, el borrado se hace igual.

\end{itemize} 
\end{itemize}
\begin{figure}
		\centering
		\includegraphics[width=0.75\textwidth] {rutas.jpg}
		\caption{Casos de uso de gestión de rutas }
		\label{fig:rutas}
	\end{figure}


\section{Análisis de riesgos}

Un aspecto fundamental que debe gestionarse a lo largo del desarrollo de un proyecto
son los riesgos. Mediante un tratamiento de la incertidumbre podremos evitar
las posibles la exposición y el impacto.

\subsection{Riesgos identificados}
\begin{figure}[H]
		\centering
		\includegraphics[width=0.75\textwidth] {riesgos.png}
		\caption{Identificación y clasificación de riesgos del proyecto }\label{fig:riesgos}
	\end{figure}
	
	
Una vez identificados y clasificados los riesgos debemos realizar dos acciones diferentes dependiendo del riesgo que conlleven:
\subsection{Planes de contingencia} 

Los riesgos que tengan una alta exposición pueden hacer peligrar el futuro del proyecto por ello,  debemos analizarlos en fases tempranas, antes
de comenzar el diseño o la implementación o vigilarlos. En el caso de este proyecto la manera de solventarlos fue diferente.  A continuación se comenta el R1.

 

\begin{itemize}
\item \textbf{R1} \textbf{- Curva aprendizaje Android}. Este riesgo era algo que se conocía antes de comenzar ya con el proyecto. Por ello era algo que se tenía  muy en cuenta. Para atenuar este riesgo antes de comenzar  con el proyecto se realizaron una serie de cursos on-line en \textit{Pluralsight} ya que era la manera de menguar esta curva de aprendizaje.






	
\end{itemize}

\subsection{Seguimiento} 	
	Para el resto de riesgos se hará el seguimiento.
	\begin{itemize}
	\item\textbf{ R2 Precisión en la obtención de coordenadas GPS}\\
	Este riesgo venía dado por el temor a que el GPS del móvil no fuera lo suficientemente preciso en  lugares donde se podían realizar rutas. Esto era algo comprensible ya que los GPS del los móviles no son muy muy precisos y en el caso de pintar la rutas seguidas podían producir algún que otro error. Un problema encontrado fue que las coordenadas cambiaban repentinamente sin que el usuario se desplazara tan rápido. Este problema fue solventado quitando las coordenadas que variaran demasiado en un corto periodo de tiempo, esta solución será comentada en el capítulo de implementación de la aplicación.


\item \textbf{R3 Adaptación de la metodología Scrum en el proyecto}\\
Antes de la realización de este proyecto mi conocimiento sobre la metodología Scrum era bastante escaso. Para solucionar esto antes del comienzo del proyecto hubo un periodo de familiarización con esta metodología. 

	\end{itemize}
\newpage
\section{Interfaz y usabilidad}
Una vez obtenidos los casos de uso, comenzamos creando las maquetas de cómo debería que ser a interface del usuario.
Se busca que sea una interfaz simple  intuitiva.
Para hacer esto posible seguiremos las pautas y usaremos los elementos de Material Design \cite{1}. 


Material Design es un lenguaje de diseño para distintas plataformas y dispositivos. Ofrece una guía que se debe seguir para que los usuarios  tengan una experiencia común y habitual entre distintas aplicaciones.


	
	
	
	\begin{figure}[htbp]
\begin{minipage}[b]{0.5\linewidth} %Una minipágina que cubre la mitad de la página
\centering
\includegraphics[width=6cm]{maqueta/Iniciar.png}
 
\caption{Inisiar sesión}
\label{fig:iniciar}
\end{minipage}
\hspace{0.5cm} % Si queremos tener un poco de espacio entre las dos figuras
\begin{minipage}[b]{0.5\linewidth}
\centering
\includegraphics[width=6cm]{maqueta/Registrarse.png}
 
\caption{Registrar usuario}
\label{fig:regis}
\end{minipage}

\end{figure}

En la Figura \ref{fig:iniciar}  podemos ver la maqueta para el inicio de la sesión en la que se piden los datos del email y la contraseña, mientras que en la Figura \ref{fig:regis} vemos los datos necesarios para registrar al usuario en la aplicación.







	
	\begin{figure}[htbp]
\begin{minipage}[b]{0.5\linewidth} %Una minipágina que cubre la mitad de la página
\centering
\includegraphics[width=6cm]{maqueta/Crear-Grupo.png}
\caption{Crear grupo}
\label{fig:crearg}
\end{minipage}
\hspace{0.5cm} % Si queremos tener un poco de espacio entre las dos figuras
\begin{minipage}[b]{0.5\linewidth}
\centering
\includegraphics[width=6cm]{maqueta/Ver-Miembros-grupo.png}

\caption{Añadir usuarios a grupo}
\label{fig:anadir}
\end{minipage}
\end{figure}
	
En las figuras \ref{fig:crearg} y en la \ref{fig:anadir} podemos ver como se crearía un grupo y después como se añadirían los usuario al grupo creado.
	
	
	
	\begin{figure}[H]
\begin{minipage}[b]{0.5\linewidth} %Una minipágina que cubre la mitad de la página
\centering
\includegraphics[width=6cm]{maqueta/lista-grupos.png}

\caption{Listar grupos}
\label{fig:verg}
\end{minipage}
\hspace{0.5cm} % Si queremos tener un poco de espacio entre las dos figuras
\begin{minipage}[b]{0.5\linewidth}
\centering
\includegraphics[width=6cm]{maqueta/opciones.png}
 
\caption{Opciones generales}
\label{fig:opc}
\end{minipage}
\end{figure}
	

En la figura \ref{fig:verg} vemos como se mostrarían los grupos a los que pertenece el usuario y a continuación vemos la maqueta de como serían las opciones que tiene el usuario para interactuar con la aplicación, como vemos en la figura  \ref{fig:opc}.













	\begin{figure}[H]
\begin{minipage}[b]{0.5\linewidth} %Una minipágina que cubre la mitad de la página
\centering
\includegraphics[width=6cm]{maqueta/pdi1.png}
 
\caption{Crear PDI}
\label{fig:pdi1}
\end{minipage}
\hspace{0.5cm} % Si queremos tener un poco de espacio entre las dos figuras
\begin{minipage}[b]{0.5\linewidth}
\centering
\includegraphics[width=6cm]{maqueta/pdi2.png}

\caption{Visualizar PDIs}
\label{fig:pdi2}
\end{minipage}
\end{figure}
	
 A continuación mostraríamos como se crearía un PDI en la figura  \ref{fig:pdi1} y como lo veríamos una vez creado la figura \ref{fig:pdi2}.	

	\begin{figure}[H]
\begin{minipage}[b]{0.5\linewidth} %Una minipágina que cubre la mitad de la página
\centering
\includegraphics[width=6cm]{maqueta/Trayecto-actual.png}

\caption{Ruta individual}
\label{fig:ind}
\end{minipage}
\hspace{0.5cm} % Si queremos tener un poco de espacio entre las dos figuras
\begin{minipage}[b]{0.5\linewidth}
\centering
\includegraphics[width=6cm]{maqueta/Trayecto-actual-compartido.png}

\caption{Ruta compartida}
\label{fig:comp}
\end{minipage}
\end{figure}

Por último vemos las maquetas de como seria una ruta individual, como podemos ver en la figura \ref{fig:ind} y como seria la compartida en la figura \ref{fig:comp}.