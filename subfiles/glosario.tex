
\renewcommand*{\arraystretch}{1.5}
\begin{description}
\item{ACID} \textit{Atomicity, Consistency, Isolation and Durability}: es un conjunto de propiedades que garantizan la fiabilidad de las transacciones de una base de datos. La atomicidad implica que cada transacción se ejecuta de forma completa o no se ejecuta. La consistencia garantiza que cada transacción llevará la base de datos de un estado consistente a otro consistente. El aislamiento permite que parezca que las transacciones se ejecutan de forma serializada cuando en realidad existe concurrencia. Por último, la durabilidad garantiza que una vez finalizada con éxito una transacción sus cambios se guardarán de forma persistente.

\item{AM} \textit{Agile Modeling}: una metodología complementaria para modelar y documentar sistemas software mediante una serie de buenas prácticas basadas en el desarrollo ágil.

\item {Big Data} Es el término que se utiliza para llamar a los sistemas que procesan grandes colecciones de datos que son inmanejables bajo las herramientas tradicionales.

\item{CF} \textit{Collaborative Filtering}: los algoritmos de filtrado colaborativo son una familia de algoritmos de recomendación cuyo funcionamiento se basa en generar sugerencias basándose en las preferencias de otros usuarios.

\item{HDFS} \textit{Hadoop Distributed File System}: es un sistema de fichero distribuido, escalable y portable escrito en java y diseñado para servir de forma de almacenamiento a Hadoop.

\item{IR} \textit{Information Retrieval}: la recuperación de información es la ciencia que estudia la obtención de información relevante a partir de una colección de documentos.

\item{ML} \textit{Machine Learning}: el aprendizaje automático es una rama de la inteligencia artificial que busca construir y estudiar sistemas que sean capaces de aprender de los datos.

\item{MVC} \textit{Model-view-controller}: es un patrón arquitéctonico para implementar interfaces de usuario. Divide el software en tres componentes. El modelo es el encargado de gestionar los datos y la lógica de negocio. La vista se encarga de la representación gráfica de la información. Por último, el controlador es el encargado de poner en comunicación los otros componentes.

\item{NoSQL} \textit{Not Only SQL}: clase de sistemas de gestión de bases de datos que difieren del modelo relacional y cuyo principal objetivo es proporcional escalabilidad horizontal.

\item{ORM} \textit{Object-Relational Mapping}: El mapeo objeto-relacional es una herramienta de programación que establece una correspondencia entre clases y objetos de un lenguaje de programación orientado a objetos con tablas y filas de una base de datos relacional siguiendo el patrón arquitectónico \textit{active record}.

\item{RDBMS} \textit{Relational DataBase Management System}: un sistema de gestión de base de datos relacional es un software diseñado para gestionar la creación, modificación y consulta de bases de datos bajo el esquema relacional.

\item{RecSys} \textit{Recommender Systems}: los sistemas de recomendación tienen como objetivo predecir la preferencia de un usuario hacia un producto.

\item{RM} \textit{Relevance Models}: forma abreviada empleada en esta memoria referirse a los \textit{Relevance-based Language Models}, una técnica de IR que introduce el concepto de relevancia en los modelos de lenguaje estadísticos.

\item{RPC} \textit{Remote Procedure Call}: es un tipo de comunicación entre procesos que permite la ejecución de una rutina en un equipo remoto.

\item{RUP} \textit{Rational Unified Process}: se trata de la propuesta de IBM para implementar una metodología de desarrollo basada en el marco que establece el Proceso Unificado.

\item{SQALE} \textit{Software Quality Assessment based on Lifecycle Expectations}: es un método genérico para evaluar la calidad de un código mediante unos índices y unos indicadores.

\item{WSGI} \textit{Web Server Gateway Interface}: es una interfaz estándar de Python para la comunicación entre servidores y aplicaciones web.
\end{description}