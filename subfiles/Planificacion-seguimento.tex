

 En este capítulo se comentará el proceso seguido para la realización de este proyecto detallando la planificación, diseño y el desarrollo del mismo estructurado en etapas.
 
\section{Planificación}

Una vez hecho el análisis, conocidos los objetivos que se quieren alcanzar en este proyecto y las herramientas que se utilizarán  empezaremos con la planificación del mismo.

Un punto muy  importante a tener en cuenta a la hora de realizar una planificación sería conocer los recursos y el tiempo del que disponemos.




Éste punto tan crucial se acentúa ya que este proyecto será hecho por una única persona lo que producirá sobrecargas en la planificación. Esto normalmente no es así ya que los proyectos se realizan por grupos de trabajos y muchas tareas pueden realizarse en paralelo disminuyendo apreciablemente la duración del proyecto.

Para realizar la planificación del proyecto hemos dividido el proyecto en tareas con unos objetivos bien marcados. A su vez al seguir la metodología Scrum dividimos el proyecto en etapas de tiempo  donde incluimos las tareas a realizar.
En todas estas etapas incluimos los siguientes pasos:
 
 


\begin{itemize}
\item Creación de la lista de las tareas que vamos a realizar en esta etapa.
\item  División de las tareas en tareas mas sencillas y mas fáciles de manejar.



\item Diseño de cada una de las tareas.
\item  Implementacion.
\item Pruebas
\end{itemize}

\section{Cifras generales}
No cadro de a continuacion detallanse as estatsticas do desenvolvemento da aplicacion,
o numero de clases totales Java e o numero de linas de codigo.


A aplicacion servidor conta cun total de 33 clases de Java. Contando todas elas cun
numero de linas de codigo de 1651.
Como podemos observar a aplicacion mobil e moito mais extensa e completa. Contando
cun total de 61 clases Java e 12423 linas de codigo sen saltos de lina nin comentarios. Estas
estatsticas non tenen en conta os arquivos XML de dese~no das vistas e conguracion de
Android que forman un total de 47 arquivos.


\section{Coste}
O
realizar unha avaliacion dos custos dun proxecto deberase ter en conta tanto os gastos
en persoal, como tamen os gastos en materiais e licenzas de uso. Neste caso obviaremos os
gastos en material xa que o alumno contaba con todo o necesario. Non houbo tampouco
gasto en licenzas xa que a maiora das ferramentas empregadas son libres e non e necesario
o pago polo seu uso ou ben ofrecen licenzas de balde a estudantes.


Polo tanto, terase en conta so os gastos de recursos humanos, que os forman dous
directores de proxectos e un analista/programador.


\begin{itemize}
\item Directores de proxecto son as persoas encargadas da toma de decision sobre a
supervision e avaliacion do proxecto.

\item Analista/programador e o encargado das tarefas de analise e deseno como tamen
das de tarefas de desenvolvemento e realizacion das probas do sistema.


Asumiremos, para a estimacion dos custos, un salario de 40 e/hora para un director
de proxecto e un salario dun analista/programador de 27 e/hora. Tendo en conta unha
xornada laboral de 6 horas e que a duracion de cada Sprint e de 15 das podemos estimar
o custo total do proxecto en:
\end{itemize}
\section{Seguimiento}
Como se comentou na seccion de metodoloxa o proxecto realzase en Sprints. Denimos
un total de oito Sprints, cunha duracion xa de 15 das, no que se atopan denidos
as tarefas a realizar. A continuacion detallanse os Sprints que se realizaron o longo do
proxecto.