\documentclass[a4paper,11pt, svgnames]{article}

\usepackage[T1]{fontenc}
\usepackage[utf8]{inputenc}
\usepackage[french]{babel}
\usepackage{listings}
\usepackage{../tikz-uml}

\textwidth 18.5cm
\textheight 25.5cm
\hoffset=-2.9cm
\voffset=-2.9cm

\sloppy
\hyphenpenalty 10000000

\date{}
\title{}
\author{}

\lstdefinelanguage{tikzuml}{language=[LaTeX]TeX, classoffset=0, morekeywords={umlbasiccomponent, umlprovidedinterface, umlrequiredinterface, umldelegateconnector, umlassemblyconnector, umlVHVassemblyconnector, umlHVHassemblyconnector, umlnote, umlusecase, umlactor, umlinherit, umlassoc, umlVHextend, umlinclude, umlstateinitial, umlbasicstate, umltrans, umlstatefinal, umlVHtrans, umlHVtrans, umldatabase, umlmulti, umlobject, umlfpart, umlcreatecall, umlclass, umlvirt, umlunicompo, umlimport, umlaggreg}, keywordstyle=\color{DarkBlue}, classoffset=1, morekeywords={umlcomponent, umlsystem, umlstate, umlseqdiag, umlcall, umlcallself, umlfragment, umlpackage}, keywordstyle=\color{DarkRed}, classoffset=0,  sensitive=true, morecomment=[l]{\%}}

\begin{document}

\maketitle

If you want to copy and paste the following source code, please take care of white spaces and special characters such as the minus symbol !

\medskip
\lstset{breaklines=true, frame=trBL, language=tikzuml}
\begin{lstlisting}
\begin{umlseqdiag}
\umlactor[class=A]{a}
\umldatabase[class=B, fill=blue!20]{b}
\umlmulti[class=C]{c}
\umlobject[class=D]{d}
\begin{umlcall}[op=opa(), type=synchron, return=0]{a}{b}
\begin{umlfragment}
\begin{umlcall}[op=opb(), type=synchron, return=1]{b}{c}
\begin{umlfragment}[type=alt, label=condition, inner xsep=8, fill=green!10]
\begin{umlcall}[op=opc(), type=asynchron, fill=red!10]{c}{d}
\end{umlcall}
\begin{umlcall}[type=return]{c}{b}
\end{umlcall}
\umlfpart[default]
\begin{umlcall}[op=opd(), type=synchron, return=3]{c}{d}
\end{umlcall}
\end{umlfragment}
\end{umlcall}
\end{umlfragment}
\begin{umlfragment}
\begin{umlcallself}[op=ope(), type=synchron, return=4]{b}
\begin{umlfragment}[type=assert]
\begin{umlcall}[op=opf(), type=synchron, return=5]{b}{c}
\end{umlcall}
\end{umlfragment}
\end{umlcallself}
\end{umlfragment}
\end{umlcall}
\umlcreatecall[class=E, x=8]{a}{e}
\begin{umlfragment}
\begin{umlcall}[op=opg(), name=test, type=synchron, return=6, dt=7, fill=red!10]{a}{e}
\umlcreatecall[class=F, stereo=boundary, x=12]{e}{f}
\end{umlcall}
\begin{umlcall}[op=oph(), type=synchron, return=7]{a}{e}
\end{umlcall}
\end{umlfragment}
\end{umlseqdiag}
\end{lstlisting}

\begin{center}
\begin{tikzpicture}
\begin{umlseqdiag}
\umlactor[class=A]{a}
\umldatabase[class=B, fill=blue!20]{b}
\umlmulti[class=C]{c}
\umlobject[class=D]{d}
\begin{umlcall}[op=opa(), type=synchron, return=0]{a}{b}
\begin{umlfragment}
\begin{umlcall}[op=opb(), type=synchron, return=1]{b}{c}
\begin{umlfragment}[type=alt, label=condition, inner xsep=8, fill=green!10]
\begin{umlcall}[op=opc(), type=asynchron, fill=red!10]{c}{d}
\end{umlcall}
\begin{umlcall}[type=return]{c}{b}
\end{umlcall}
\umlfpart[default]
\begin{umlcall}[op=opd(), type=synchron, return=3]{c}{d}
\end{umlcall}
\end{umlfragment}
\end{umlcall}
\end{umlfragment}
\begin{umlfragment}
\begin{umlcallself}[op=ope(), type=synchron, return=4]{b}
\begin{umlfragment}[type=assert]
\begin{umlcall}[op=opf(), type=synchron, return=5]{b}{c}
\end{umlcall}
\end{umlfragment}
\end{umlcallself}
\end{umlfragment}
\end{umlcall}
\umlcreatecall[class=E,x=8]{a}{e}
\begin{umlfragment}
\begin{umlcall}[op=opg(), name=test, type=synchron, return=6, dt=7, fill=red!10]{a}{e}
\umlcreatecall[class=F, stereo=boundary, x=12]{e}{f}
\end{umlcall}
\begin{umlcall}[op=oph(), type=synchron, return=7]{a}{e}
\end{umlcall}
\end{umlfragment}
\end{umlseqdiag}
\end{tikzpicture}

\begin{tikzpicture}
\end{tikzpicture}
\end{center}
\end{document}

